\chapter{Prefácio}

Como o título indica, este é um livro sobre lógica formal.
A lógica formal estuda relações de consequência, nas quais uma conclusão se segue de certas informações de partida, as premissas.
O aspecto formal advém do emprego de \emph{linguagens formais}, isto é, um simbolismo precisamente definido.
Para estas linguagens formais é possível definir rigorosamente a relação de consequência.
Uma definição de consequência juntamente com a linguagem formal respectiva compõem um \emph{sistema lógico}.
Os sistemas lógicos, ou simplesmente \emph{lógicas}, são produtos de uma investigação que ocupou-se em analisar e modelar práticas inferenciais envolvendo as noções de conjunção, disjunção, negação, condicional, necessidade, obrigatoriedade, provabilidade, crença, tempo, entre muitas outras.
Este livro apresentará lógicas que exploram algumas destas noções.


A parte~\ref{ch.intro} introduz o assunto e as noções da lógica de maneira informal, ainda sem utilizar uma linguagem formal.
As partes \ref{ch.TFL}, \ref{ch.TruthTables} e \ref{ch.NDTFL} tratam da lógica proposicional.  Nesta lógica,  se estuda certos conectivos, os quais corrempendem, no português, às expressões tais como `ou', `e', `não', `se \dots então'.
A relação de consequência é discutida de duas maneiras:
semanticamente, usando o método das tabelas de verdade (na Parte~\ref{ch.TruthTables}) e demonstrativamente, usando um sistema de derivações formais (na Parte~\ref{ch.NDTFL}).
As partes \ref{ch.FOL}, \ref{ch.semantics} e \ref{ch.NDFOL} lidam com uma lógica mais complexa, a lógica de primeira ordem.
Além dos conectivos da lógica proposicional, a linguagem da lógica de primeira ordem inclui também nomes, predicados, a relação de identidade e os quantificadores.
Esses elementos adicionais da linguagem a tornam muito mais expressiva do que a linguagem proposicional, e passaremos um bom tempo investigando o quanto se pode expressar nela.
As noções da lógica de primeira ordem também são definidas tanto semanticamente, através de interpretações (na Parte \ref{ch.semantics}), quanto demonstrativamente (na Parte \ref{ch.NDFOL}), usando uma versão mais complexa do sistema de derivação formal introduzido na Parte~\ref{ch.NDTFL}.
A Parte \ref{ch.ML} discute uma extensão da LPC (a lógica proposicional) obtida a partir de operadores não verofuncionais para a possibilidade e a necessidade, conhecida como lógica modal.
A Parte \ref{ch.normalform} abrange dois tópicos avançados:
o tópico das formas normais (conjuntiva e disjuntiva) e da adequação expressiva dos conectivos proposicionais, e o tópico da correção do sistema de dedução natural para a LPC.

Nos apêndices, você encontrará uma discussão sobre notações alternativas para as linguagens tratadas neste texto, uma outra sobre sistemas de derivação alternativos, além  de um guia de referência rápida listando a maioria das regras e definições importantes.
Os termos principais estão listados em um glossário no final.

Você é livre para copiar e redistribuir gratuitamente este material em qualquer meio ou formato, remixar, transformar e desenvolvê-lo para qualquer finalidade, mesmo comercialmente, desde que respeite as restrições da licença  \href{https://creativecommons.org/licenses/by/4.0/}{Creative Commons Attribution 4.0}.
